% -*- coding: utf-8 -*-
\documentclass[a4paper]{article}
\usepackage{listings}
\usepackage{hyperref}
\usepackage{makeidx}
\usepackage[totoc]{idxlayout}

\lstMakeShortInline[basicstyle=\small\ttfamily]|

\makeindex

\begin{document}
\title{Pyphant Manual}
\author{Alexander Held}

\maketitle

\date{}

\begin{abstract}
  This manual describes the structure and use of pyphant (TODO:
  Citation). Pyphant is an open source project currently developed
  mainly by the service group ``Scientific Data Processing'' at the
  Freiburg Materials Research Center, University of Freiburg,
  Germany. Pyphant consists of a collection of python packages
  offering a framework for scientific data analysis including a
  collection of building blocks organized into toolboxes. Pyphant
  features a flexible plugin architecture allowing for the application
  in many different fields. It ships with a graphical user interface
  allowing for the development of scientific data analysis workflows
  using graphical programming.
\end{abstract}

\tableofcontents

\section{Introduction}
\label{sec:introduction}

\subsection{Installation}
\label{sec:introduction_installation}

\subsection{Pyphant Quick Tour}
\label{sec:introduction_a_quick_tour}

\section{Data Model}
\label{sec:data_model}

\subsection{Quantities}
\label{sec:data_model_quantities}

\subsection{Field Containers}
\label{sec:data_model_fcs}

\subsection{Sample Containers}
\label{sec:data_model_scs}

\subsection{Knowledge Manager}
\label{sec:data_model_knowledge_manager}

\section{Data Processing}
\label{sec:data_processing}

\subsection{Workers}
\label{sec:data_processing_workers}

\subsection{Recipes}
\label{sec:data_processing_recipes}

\subsection{Visualizers}
\label{sec:data_processing_visualizers}

\bibliography{bibliography}

\clearpage

\printindex

\end{document}

%%% Local Variables:
%%% TeX-PDF-mode: t
%%% End:
